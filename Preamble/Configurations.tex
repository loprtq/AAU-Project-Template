% Preamble/Configurations.tex
% -------------------------------------------------------------------------------
% Configurations for the report


% Basic configurations ----------------------------------------------------------
\raggedbottom


% Nomenclature ------------------------------------------------------------------
\renewcommand{\nomname}{Nomenclature}                   % Title name
\setlength{\nomlabelwidth}{1.5em}                       % Text to symbol distance





% Custom colors -----------------------------------------------------------------
% Official AAU colors
%https://www.design.aau.dk/om-aau-design/Farver/
\definecolor{AAUblue1}{RGB}{ 33, 26, 82}
\definecolor{AAUblue2}{RGB}{ 89, 79,191}
\definecolor{AAUgrey1}{RGB}{ 84, 97,110}
\definecolor{AAUgrey2}{RGB}{104,119,132}
\definecolor{AAUgrey3}{RGB}{162,172,182}
\definecolor{AAUgrey4}{RGB}{222,223,226}
\definecolor{AAUgrey5}{RGB}{241,242,243}
\definecolor{AAUgrey6}{RGB}{248,248,249}
\definecolor{AAUgreen}{RGB}{157,187, 29}
\definecolor{AAUteal} {RGB}{ 94,150,149}
\definecolor{AAUred}  {RGB}{223,103, 82}





% Header and footer -------------------------------------------------------------
% http://mirrors.dotsrc.org/ctan/macros/latex/contrib/fancyhdr/fancyhdr.pdf
\setlength{\headheight}{15pt}                           % Headheight

\fancyhead{}                                            % Clears header
\fancyhead[LE,RO]{Group \projectgroup}                  % Left on even, right on odd pages
\fancyhead[RE]{\leftmark}                               % Right on even pages
\fancyhead[LO]{\rightmark}                              % Left on odd pages

\fancyfoot{}                                            % Clears footer
\fancyfoot[CE,CO]{\thepage}                             % Centered page numbers
\pagestyle{fancy}                                       % Activates fancy pages





% Heading -----------------------------------------------------------------------
% http://mirrors.dotsrc.org/ctan/macros/latex/contrib/titlesec/titlesec.pdf
% Dele skrives på to linjer
\titleformat{\part}[display]
 {\Huge\bfseries\color{AAUblue1}\filcenter}
 {\huge\partname{} \thepart}
 {0em}
 {}
 
\titlespacing*{\chapter}{0pt}{0.25em}{1em}[0in]
\newlength\Tlen\setlength\Tlen{\evensidemargin+0.5in+\hoffset}
\titleformat{\chapter} % Title with margin numbering
  {\huge\color{AAUblue1}\bfseries}
  {\llap{\hspace*{-\Tlen}\thechapter\hfill}}
  {0em}
  {\huge\bfseries\color{AAUblue1}}
% \titleformat{\chapter}[hang] % Title with normal numbering
%  {\huge\bfseries\color{AAUblue1}}
%  {\thechapter\hspace{0.75em}\textcolor{AAUgrey3}{$\mid$}}
%  {0.75em}
%  {\huge\bfseries\color{AAUblue1}}

\titleformat{\section}[hang]
 {\Large\bfseries}
 {\thesection}
 {1em}
 {\Large\bfseries}
 
\titleformat{\subsection}[hang]
 {\large\bfseries}
 {\thesubsection}
 {1em}
 {\large\bfseries}

% Adjust title and table of contents with regards to the appendix
\renewcommand{\appendixname}{Appendix}
\renewcommand{\appendixtocname}{Appendices}
\renewcommand{\appendixpagename}{\textcolor{AAUblue1}{Appendices}}
\xpatchcmd{\addappheadtotoc}{{chapter}}{{part}}{}{}





% Figure settings ---------------------------------------------------------------
% http://mirrors.dotsrc.org/ctan/macros/latex/contrib/caption/caption-eng.pdf
\captionsetup{font=it,labelfont={bf,sc}}            % Italic text and bold label

% https://tex.stackexchange.com/q/42611
\usetikzlibrary{arrows.meta}                        % Different arrowheads
\usetikzlibrary{calc}                               % Better calculations
\usetikzlibrary{positioning}                        % Better positioning

% Predefined styles
\tikzstyle{point} = [fill,shape=circle,minimum size=3pt,inner sep=0pt]
\tikzstyle{edge}  = [fill=white,midway,inner sep=1pt]



% Algorithms --------------------------------------------------------------------
% Settings for algorithms
\floatname{algorithm}{Algorithm}
\renewcommand{\listalgorithmname}{List of Algorithms}





% Kildekode ---------------------------------------------------------------------
% Syntax highlighting for Python-code
% https://en.wikibooks.org/wiki/LaTeX/Source_Code_Listings
\lstdefinestyle{custompy}{
    belowcaptionskip=\baselineskip,
    breaklines=true,
    mathescape=true,
    escapeinside={@}{@},
    inputencoding=utf8,
    extendedchars=true,
    language=Python,
    showstringspaces=false,
    basicstyle=\footnotesize\ttfamily,
    keywordstyle=\bfseries\color{AAUblue1},
    commentstyle=\itshape\color{AAUgrey3},
    identifierstyle=\color{AAUblue2},
    stringstyle=\color{AAUteal},
}
\lstdefinestyle{customR}{
    belowcaptionskip=\baselineskip,
    breaklines=true,
    mathescape=true,
    escapeinside={@}{@},
    inputencoding=utf8,
    extendedchars=true,
    language=R,
    showstringspaces=false,
    basicstyle=\footnotesize\ttfamily,
    keywordstyle=\bfseries\color{AAUblue2},
    commentstyle=\itshape\color{AAUgrey3},
    identifierstyle=\color{AAUblue1},
    stringstyle=\color{AAUteal},
}
\lstdefinestyle{customJ}{
    belowcaptionskip=\baselineskip,
    breaklines=true,
    mathescape=true,
    escapeinside={@}{@},
    inputencoding=utf8,
    extendedchars=true,
    language=Julia,
    showstringspaces=false,
    basicstyle=\footnotesize\ttfamily,
    keywordstyle=\bfseries\color{AAUblue2},
    commentstyle=\itshape\color{AAUgrey3},
    identifierstyle=\color{AAUblue1},
    stringstyle=\color{AAUteal},
}

\lstset{style=custompy, captionpos=b, frame=lines, rulecolor=\color{AAUblue1}}
\renewcommand{\lstlistingname}{Listing}
\providecommand*{\listingautorefname}{Listing}
\renewcommand{\lstlistlistingname}{List of Listings}





% Setting up autoref ------------------------------------------------------------
\def\partautorefname{Part}
\def\chapterautorefname{Chapter}
\def\sectionautorefname{Section}
\def\subsectionautorefname{Subsection}

\def\figureautorefname{Figure}
\def\tableautorefname{Table}
\def\algorithmautorefname{Algorithm}
\def\lstinputlistingautorefname{Listing}

\def\defiautorefname{Definition}
\def\theoautorefname{Theorem}
\def\lemmautorefname{Lemma}
\def\corolautorefname{Corollary}
\def\examautorefname{Example}
\def\propoautorefname{Proposition}





% Custom environments -----------------------------------------------------------
% Unifies the counters in the report
\makeatletter\let\c@lemm    = \c@theo\makeatother
\makeatletter\let\c@corol   = \c@theo\makeatother
\makeatletter\let\c@defi    = \c@theo\makeatother
\makeatletter\let\c@exam    = \c@theo\makeatother
\makeatletter\let\c@propo   = \c@theo\makeatother

% Environment for theorems, propositions, corollary
\newmdenv[
    leftmargin          = 0em,
    innerleftmargin     = 1em,
    innertopmargin      = 0.5em,
    innerbottommargin   = 1em,
    innerrightmargin    = 1em,
    rightmargin         = 0em,
    linewidth           = 3pt,
    topline             = false,
    rightline           = false,
    bottomline          = false,
    skipabove           = 0.75em,
    skipbelow           = 0.75em,
    nobreak             = true,
]{thmenv}

% Environment for definitions
\newmdenv[
    leftmargin          = 0em,
    innerleftmargin     = 1em,
    innertopmargin      = 0.5em,
    innerbottommargin   = 1em,
    innerrightmargin    = 1em,
    skipabove           = 0.75em,
    skipbelow           = 0.75em,
    rightmargin         = 0em,
    linewidth           = 3pt,
    linecolor           = AAUblue1,
    topline             = false,
    rightline           = false,
    bottomline          = false,
    backgroundcolor     = AAUgrey5,
    nobreak             = true,
]{defnenv}
  
% Environment for examples
\newmdenv[
    leftmargin          = 0em,
    innerleftmargin     = 1em,
    innertopmargin      = 0em,
    innerbottommargin   = 1em,
    innerrightmargin    = 1em,
    rightmargin         = 0em,
    skipabove           = 1.5em,
    skipbelow           = 1.5em,
    linewidth           = 1pt,
    linecolor           = AAUblue1,
    topline             = true,
    rightline           = false,
    leftline            = false,
    bottomline          = true,
    % nobreak             = true,
    backgroundcolor     = AAUgrey6,
    frametitle          = {}
]{exmpenv}


% Links and references  ---------------------------------------------------------
% https://tex.stackexchange.com/questions/50747/options-for-appearance-of-links-in-hyperref
\newcommand{\email}[1]{$\left<\right.$\href{mailto:#1}{#1}$\left.\right>$}
\PassOptionsToPackage{hyphens}{url}
\renewcommand{\UrlFont}{\itshape\ttfamily\normalfont}
\hypersetup{
    colorlinks  = true,                             % 
    breaklinks  = true,                             % 
    linkcolor   = black,                            % Black links
    filecolor   = black,                            % Black file references
    citecolor   = black,                            % Black citations
    urlcolor    = AAUblue2                          % AAUblue2 urls
}





% Warnings ----------------------------------------------------------------------
\WarningsOff[natbib]





% PDF metadata ------------------------------------------------------------------
\hypersetup{
    pdftitle    = {\projecttitle},                  
    pdfsubject  = {\projecttheme},                  
    pdfkeywords = {Jonas Benjamin Rose Aldous}{\projecttheme, \projecttitle, \projectsubtitle, English, Aalborg University, Aalborg Universitet, Matematik-Økonomi, Mathematics-Economics, \projectnumber, \projectperiod},
    pdfauthor   = {Jonas Benjamin Rose Aldous},     
    pdfproducer = {Jonas Benjamin Rose Aldous},     
    pdfstartview= {FitH},                           % Fits the width of the page to the window
}